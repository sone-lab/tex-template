%! TEX root = ../readme_report.tex
\documentclass[../readme_report]{subfiles}

\begin{document}

\MyChapter{卒業論文を書くときの注意点}

\section{卒業論文とは何か}

卒業論文は,自分が行った研究を,第三者に正確に説明するための学術文書です.

\begin{itemize}
    \item 感想文やレポートではありません
    \item 「頑張った」「難しかった」などの主観的記述は不要です
\end{itemize}

読んだ人が以下を理解できることが重要です.

\begin{itemize}
    \item 何を目的に
    \item 何をして
    \item どんな結果が出て
    \item それをどう解釈したのか
\end{itemize}

\section{論文全体の構成について}

基本的には,以下の流れが自然になるように構成してください.

\subsection{構成の要素}

\begin{enumerate}
    \item \textbf{背景・目的}:なぜこの研究を行ったのか
    \item \textbf{方法}:何を,どのように行ったのか
    \item \textbf{結果}:実験・解析で何が得られたのか
    \item \textbf{考察}:結果が何を意味するのか
    \item \textbf{結論}:研究として何が分かったのか
\end{enumerate}

章や節の順番は,「初めて読む人が迷わないか?」を基準に考えてください.

\section{図・表・数式の使い方}

\subsection{図・表について}

図や表は本文中で必ず説明してください.

\begin{itemize}
    \item \textbf{NG:}「下図に示す」
    \item \textbf{OK:} \Fig{fig:example}に示すように,\ldots
\end{itemize}

図や表は単体で見ても意味が分かるように作ります.以下を確認してください.

\begin{itemize}
    \item 軸の名前
    \item 単位
    \item 記号の説明
\end{itemize}

「見た目がきれい」より「情報が正確」が重要です.

\subsection{数式について}

数式を出すときは,\textbf{なぜその式を使うのか}を文章で説明してください.

\begin{itemize}
    \item いきなり式だけを出すのはNGです
    \item 数式も,本文中で番号を使って参照してください
\end{itemize}

例えば,\Eq{eq:example}のような形式で参照します.

\section{図の作成に関する注意}

\begin{itemize}
    \item ExcelやPowerPointで作った図を使っても構いません
    \item ただし,\textbf{文字が小さすぎて読めない図はNG}です
    \item 不要な色や装飾は避け,説明に必要な情報だけを残してください
    \item 図は「飾り」ではなく「説明のための道具」です
\end{itemize}

\section{文章の書き方について}

\begin{itemize}
    \item 文体は \textbf{「だ・である調」}に統一してください
    \item 主語と述語が対応しているか,必ず見直してください
    \item 以下の表現は使いすぎないよう注意してください
          \begin{itemize}
              \item 「\(\sim \)と思われる」
              \item 「\(\sim \)と考えられる」
          \end{itemize}
    \item 事実(結果)と考察(解釈)は,はっきり区別して書きましょう
    \item 箇条書きは便利ですが,多用せず,文章で説明することを基本としてください
\end{itemize}

\section{引用・参考文献について}

\begin{itemize}
    \item 他人の研究成果や文章を使う場合は,\textbf{必ず引用}してください
    \item 図や表を参考にした場合も,出典を明記します
    \item Webページも参考文献として記載が必要です
    \item 「自分の結果」と「他人の成果」が混ざらないよう注意してください
\end{itemize}

本文中での引用例は,\cite{ref:example}のように記述します.

\section{再現性を意識する}

卒業論文では,\textbf{他人が同じことをすれば,同じ結果が出るか}が重要です.

以下の情報は,省略せずに具体的に書いてください.

\begin{itemize}
    \item 実験条件
    \item 使用した装置
    \item パラメータ設定
\end{itemize}

「一般的だから」「よく知られているから」と省略する場合は,どの文献や知識に基づくかを示してください.

\section{禁止事項・注意事項}

\begin{itemize}
    \item 他人の文章や図を,出典なしで使ってはいけません
    \item コピー \&ペーストの多用は不正行為になります
    \item 生成AIを使った場合も,\textbf{内容の責任は自分にあります}
    \item 分からないことがあれば,\textbf{締切直前ではなく早めに相談}してください
\end{itemize}

\section{\LaTeX での具体的な書き方}

このテンプレートを使用する際の,具体的な記述方法を説明します.

\subsection{図の挿入方法}

図を挿入する際は,\texttt{figure}環境を使用します.

\begin{equation}
    \text{配置オプション} = \text{[htbp]}
    \label{eq:placement}
\end{equation}

図のコード例を示します.

\begin{itemize}
    \item \texttt{[htbp]}は配置オプション(here, top, bottom, page)です
    \item \texttt{width=\textbackslash linewidth}で図の幅を調整(例:\texttt{0.8 \textbackslash linewidth})
    \item \texttt{\textbackslash label \{fig:label \}}でラベルを付けると,本文中で参照できます
    \item 図ファイルは \texttt{figures/}フォルダに配置してください
\end{itemize}

本文中での図の参照:

\begin{minted}{latex}
\Fig{fig:label}に示すように,...
\end{minted}

または

\begin{minted}{latex}
\Figure{fig:label}に示すように,...
\end{minted}

\begin{itemize}
    \item \verb|\Fig{...}|は「Fig. 1」のように表示されます
    \item \verb|\Figure{...}|は「Figure 1」のように表示されます
\end{itemize}

\subsection{表の挿入方法}

表を挿入する際は,\texttt{table}環境を使用します.

本文中での表の参照:

\begin{minted}{latex}
\Table{tab:label}に示すように,...
\end{minted}

\subsection{数式の書き方}

\subsubsection{インライン数式(文中の数式)}

本文中に \verb|$E = mc^2$|のように記述します.

\subsubsection{別行立て数式(番号付き)}

別行立て数式では \texttt{equation}環境を使用します.

本文中での式の参照:

\begin{minted}{latex}
\Eq{eq:label}より,...
\end{minted}

\verb|\Eq{...}|は「Eq. (1)」のように表示されます.

\subsubsection{複数行の数式}

複数行の数式は \texttt{align}環境を使用します.

\begin{align}
    a & = b + c \\
    d & = e + f
\end{align}

\subsection{参考文献の引用方法}

このテンプレートでは \textbf{biblatex}を使用しています.

\subsubsection{文献情報の登録}

\verb|references.bib|ファイルに文献情報を記述します.例を示します.

本文中での引用:

\begin{minted}{latex}
先行研究 \cite{ref:yamada2020}によると,...
\end{minted}

複数の文献を同時に引用する場合:

\begin{minted}{latex}
これらの研究 \cite{ref:yamada2020, ref:tanaka2021}では,...
\end{minted}

\subsection{章・節の作成方法}

このテンプレートでは,各章を個別のファイルに分けて記述します.

\subsubsection{章の作成}

\begin{minted}{latex}
\MyChapter{章のタイトル}
\end{minted}

\begin{itemize}
    \item 通常の \verb|\chapter{...}|の代わりに \verb|\MyChapter{...}|を使用
    \item ヘッダーに章タイトルが表示されます
\end{itemize}

\subsubsection{節・小節の作成}

\begin{itemize}
    \item \texttt{\textbackslash section \{節のタイトル \}}
    \item \texttt{\textbackslash subsection \{小節のタイトル \}}
    \item \texttt{\textbackslash subsubsection \{小小節のタイトル \}}
\end{itemize}

\subsection{特殊な記号・単位の書き方}

\subsubsection{単位の記述}

このテンプレートには \texttt{siunitx}パッケージが含まれています.

\begin{minted}{latex}
温度は \SI{25}{\celsius},速度は \SI{10}{\meter \per \second} である.
\end{minted}

\subsubsection{ベクトルの記述}

\begin{minted}{latex}
ベクトル $\bm{v}$ は...
\end{minted}

\subsection{ファイル構成}

以下のファイル構成になっています.

\begin{itemize}
    \item \texttt{main.tex}:メインファイル(全体の構成を記述)
    \item \texttt{sections/}:各章のファイルを格納
          \begin{itemize}
              \item \texttt{title.tex}:タイトルページ
              \item \texttt{abstract.tex}:概要
              \item \texttt{introduction.tex}:序論
              \item \texttt{theory.tex}:理論
              \item \texttt{conclusion.tex}:結論
              \item \texttt{references.tex}:参考文献
              \item \texttt{acknowledgments.tex}:謝辞
          \end{itemize}
    \item \texttt{figures/}:図ファイルを格納
    \item \texttt{references.bib}:参考文献データベース
\end{itemize}

各章のファイルは \verb|\subfile{sections/ファイル名}|で読み込まれます.

\section{最後に}

卒業論文は「完璧さ」よりも,以下が重要です.

\begin{itemize}
    \item 読み手に伝わるか
    \item 論理の流れが通っているか
    \item 根拠が示されているか
\end{itemize}

\textbf{「これを他人が読んで理解できるか?」}を常に意識しながら書いてください.

\end{document}
