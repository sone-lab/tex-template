\documentclass{classes/resume}

\title{
	title
}

\author{
	〇〇大学 〇〇学部 〇〇学科 〇〇研究室 学籍番号 氏名
}

\begin{document}
\maketitle

\section{はじめに}
本稿では,○○について述べる.
\subsection{その次に}
本節では,○○について述べる.
\subsubsection{最後に}
最後に,○○について述べる.
\Fig{fig:fig1}に示すように,○○である.
\Table{tab:sample}に示すように,○○である.
\Eq{eq:label}に示すように,○○である.

To be, or not to be: that is the question.


% 図の挿入
\begin{figure}[htbp]
    \centering
    \includegraphics[width=\linewidth]{figures/dummy.pdf}
    \caption{図の説明 \cite{ref:nomura2022uwb} \cite{ref:青空文庫Aozo22:online}}
    \label{fig:fig1}
\end{figure}

\begin{table}[htbp]
    \centering
    \caption{Sample Table}
    \begin{tabular}{|c|c|c|}
        \hline
        1 & 2 & 3 \\
        \hline
        4 & 5 & 6 \\
        \hline
        7 & 8 & 9 \\
        \hline
    \end{tabular}
    \label{tab:sample}
\end{table}

\begin{equation}
    \left( \int_{0}^\infty \frac{\sin x}{\sqrt {x}} dx \right)^{2}
    = \sum_{k = 0}^\infty \frac{(2k)!}{2^{2k}(k!)^{2}} \frac{1}{2k + 1}
    = \prod_{k = 1}^\infty \frac{4k^{2}}{4k^{2} - 1} = \frac{\pi}{2}
    \label{eq:label}
\end{equation}

\section{結論}
結論として,○○について述べた.

\subfile{sections/references}

\end{document}