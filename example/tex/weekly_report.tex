\documentclass{classes/weekly_report}

\title{
	週報タイトル
}

\author{
	氏名
}

\meetingdate{2025年1月23日} % ミーティングの日付を設定

% -------------------------ここから-------------------------

\begin{document} 
\maketitle
\thispagestyle{preport}

% -------------------------ここまで触らない-------------------------

\section{目的}
今週の週報の目的について述べる.

\section{今週やったこと}
今週やったことについて述べる.

\subsection{やったことその一}
\label{sec:done1}
まず,○○について述べる.

\subsection{やったことその二}
次に,△△について述べる.

以下,\Fig{fig:fig1}にサンプルの図を示す.
また,\Table{tab:sample}にサンプルの表を示す.
さらに,\Eq{eq:label}にサンプルの数式を示す.

節についての専用のコマンドは用意していないが,\ref{sec:done1}節のように参照できる.
単位のある数字を記述するときは,\texttt{siunitx}パッケージを利用して,\qty{10}{\milli \meter}のように書くと良い.

% 図の挿入
\begin{figure}[htbp]
    \centering
    \includegraphics[width=\linewidth]{figures/dummy.pdf}
    \caption{図の説明 \cite{ref:nomura2022uwb} \cite{ref:青空文庫Aozo22:online}}
    \label{fig:fig1}
\end{figure}

\begin{table}[htbp]
    \centering
    \caption{Sample Table}
    \begin{tabular}{|c|c|c|}
        \hline
        1 & 2 & 3 \\
        \hline
        4 & 5 & 6 \\
        \hline
        7 & 8 & 9 \\
        \hline
    \end{tabular}
    \label{tab:sample}
\end{table}

\begin{equation}
    \left( \int_{0}^\infty \frac{\sin x}{\sqrt {x}} dx \right)^{2}
    = \sum_{k = 0}^\infty \frac{(2k)!}{2^{2k}(k!)^{2}} \frac{1}{2k + 1}
    = \prod_{k = 1}^\infty \frac{4k^{2}}{4k^{2} - 1} = \frac{\pi}{2}
    \label{eq:label}
\end{equation}

\section{まとめ}
今週のまとめを述べる.

\section{来週やること}
来週やることを述べる.

\end{document}