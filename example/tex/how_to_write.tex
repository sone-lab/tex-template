%! TEX root = ../../readme_report.tex
\documentclass[readme_report]{subfiles}

\begin{document}
\MyChapter{\LaTeX での具体的な書き方}

このテンプレートを使用する際の,具体的な記述方法を説明する.
これはあくまで一部の例であり,詳細は \LaTeX の参考書やオンラインドキュメントを参照されたい.

\section{図の挿入方法}

図を挿入する際は,\texttt{figure}環境を使用する.

図のコード例を示す.

\begin{minted}{latex}
\begin{figure}[htbp]
 \centering
 \includegraphics[keepaspectratio, width=0.8 \linewidth]{figures}
 \caption{caption}
 \label{fig:label}
\end{figure}
\end{minted}

一行ずつ解説する.

\begin{minted}{latex}
\begin{figure}[htbp]
\end{figure}
\end{minted}

\LaTeX は,何かの構文を始めるときに \texttt{\textbackslash begin \{...\}} を使い,終わるときに \texttt{\textbackslash end \{...\}} を使う.
この場合,\texttt{figure} 環境を開始および終了している.
[htbp] は,図の配置に関するオプションである.
here(ここ),top(ページの上部),bottom(ページの下部),page(専用のページ)を意味し,左から順に優先される.

\begin{minted}{latex}
 \centering
\end{minted}
このコマンドは,図を中央に配置するためのものである.

\begin{minted}{latex} 
 \includegraphics[keepaspectratio, width=0.8 \linewidth]{figures/example.pdf}
\end{minted}
このコマンドは,実際に図を挿入するためのコマンドであり,\texttt{figures/example.pdf} というファイルを挿入している.
\texttt{width=0.8 \textbackslash linewidth} は,図の幅を本文の幅の \qty{80}{\percent} に設定することを意味している.
\texttt{keepaspectratio} は,図の縦横比を維持するためのオプションである.

\begin{minted}{latex}
 \caption{caption}
\end{minted}
このコマンドは,図のキャプション(説明文)を設定するためのものである.
キャプションは図の下に表示される.

\begin{minted}{latex}
 \label{fig:label}
\end{minted}
このコマンドは,図にラベルを付けるためのものである.
ラベルを付けることで,本文中で図を参照することができる.

図のファイルは \texttt{figures/} フォルダに配置すること.


本文中での図の参照は以下のように行う.

\begin{minted}{latex}
\Fig{fig:label}に示すように,...
\end{minted}

\verb|\Fig{...}|は「Fig. 1」のように表示される

\section{表の挿入方法}

表を挿入する際は,\texttt{table}環境を使用する.

\begin{minted}{latex}
\begin{table}[htbp]
 \centering
 \caption{実験条件}
 \label{tab:conditions}
 \begin{tabular}{lcc}
 \hline
 項目 & 値 & 単位 \\
 \hline
 温度 & 25 & \si{\celsius} \\
 圧力 & 101.3 & \si{\kilo \pascal} \\
 \hline
 \end{tabular}
\end{table}
\end{minted}

一行ずつ解説する.
\begin{minted}{latex}
\begin{table}[htbp]
\end{table}
\end{minted}

\texttt{table}環境を開始および終了している.
[htbp]は,図と同様に表の配置に関するオプションである.

\begin{minted}{latex}
 \centering
\end{minted}
このコマンドは,表を中央に配置するためのものである.

\begin{minted}{latex}
 \caption{実験条件}
\end{minted}
このコマンドは,表のキャプション(説明文)を設定するためのものである.
キャプションは表の上に表示される.

\begin{minted}{latex}
 \label{tab:conditions}
\end{minted}
このコマンドは,表にラベルを付けるためのものである.
ラベルを付けることで,本文中で表を参照することができる.

\begin{minted}{latex}
 \begin{tabular}{lcc}
... \end{tabular}
\end{minted}
この部分が実際に表を作成する部分である.
\texttt{tabular}環境を使用して表を作成する.
\texttt{\{lcc \}}は,表の列の配置を指定している.
ここでは,1列目が左寄せ(l),2列目と3列目が中央寄せ(c)であることを示している.
表の内容は,行ごとに記述し,各セルは \texttt{\&}で区切る.
行の終わりは

\begin{minted}{latex}
 \\
\end{minted}
で示す.

実際に表を作成するときは,ずべて中央揃えで作成すること.
また,web上で表を作成するツールがあるため,そちらを利用することも推奨する.


\section{数式の書き方}

\subsection{インライン数式(文中の数式)}

本文中に \verb|$E = mc^2$|のように記述する.

\subsection{別行立て数式(番号付き)}

別行立て数式では \texttt{equation}環境を使用する.

\begin{equation}
    E = mc^2 \label{eq:einstein}
\end{equation}

本文中での式の参照:

\begin{minted}{latex}
\Eq{eq:einstein}より,...
\end{minted}

\verb|\Eq{...}|は「\Eq{eq:einstein}」のように表示される.

\subsection{複数行の数式}

複数行の数式は \texttt{align}環境を使用する.

\begin{align}
    a & = b + c \label{eq:first} \\
    d & = e + f \label{eq:second}
\end{align}

\begin{minted}{latex}
\begin{align}
    a & = b + c \label{eq:first} \\
    d & = e + f \label{eq:second}
\end{align}
\end{minted}

複数行の数式でも,本文中での式の参照は、\Eq{eq:first}や\Eq{eq:second}のように行うことができる.


\begin{comment}
% TODO: ここまで

\subsection{参考文献の引用方法}

このテンプレートでは \textbf{biblatex}を使用している.

\subsubsection{文献情報の登録}

\verb|references.bib|ファイルに文献情報を記述する.例を示す.

本文中での引用:

\begin{minted}{latex}
先行研究 \cite{ref:yamada2020}によると,...
\end{minted}

複数の文献を同時に引用する場合:

\begin{minted}{latex}
これらの研究 \cite{ref:yamada2020, ref:tanaka2021}では,...
\end{minted}


\subsection{章・節の作成方法}

このテンプレートでは,各章を個別のファイルに分けて記述する.

\subsubsection{章の作成}

\begin{minted}{latex}
\MyChapter{章のタイトル}
\end{minted}

\begin{itemize}
    \item 通常の \verb|\chapter{...}|の代わりに \verb|\MyChapter{...}|を使用する
    \item ヘッダーに章タイトルが表示される
\end{itemize}

\subsubsection{節・小節の作成}

\begin{itemize}
    \item \texttt{\textbackslash section \{節のタイトル \}}を使用する
    \item \texttt{\textbackslash subsection \{小節のタイトル \}}を使用する
    \item \texttt{\textbackslash subsubsection \{小小節のタイトル \}}を使用する
\end{itemize}

\subsection{特殊な記号・単位の書き方}

\subsubsection{単位の記述}

このテンプレートには \texttt{siunitx}パッケージが含まれている.

\begin{minted}{latex}
温度は \SI{25}{\celsius},速度は \SI{10}{\meter \per \second} である.
\end{minted}

\subsubsection{ベクトルの記述}

\begin{minted}{latex}
ベクトル $\bm{v}$ は...
\end{minted}

\subsection{textlintによる文章チェック}

このテンプレートには,\textbf{textlint}による日本語文章の自動チェック機能が含まれている.

\subsubsection{textlintとは}

textlintは,文章の表記ゆれや誤字,日本語の文法的な問題を自動的にチェックするツールである.

\begin{itemize}
    \item 「ですます調」と「である調」の混在をチェック
    \item 一文の長さや読点の位置をチェック
    \item 冗長な表現や不適切な表記を検出
    \item 技術文書に適した表現かどうかを検証
\end{itemize}

\subsubsection{textlintの実行方法}

ターミナルで以下のコマンドを実行する.

\begin{minted}{bash}
# 初回のみ:textlintのインストール
bash bin/install_textlint.sh

# 文章のチェック実行
bash bin/linter.sh
\end{minted}

\subsubsection{チェック結果の確認}

textlintは,問題がある箇所を以下のように報告する.

\begin{itemize}
    \item ファイル名と行番号
    \item 問題の種類(エラーまたは警告)
    \item 具体的な修正提案
\end{itemize}

\subsubsection{注意点}

\begin{itemize}
    \item textlintの指摘は \textbf{あくまで参考}である
    \item すべての指摘を修正する必要はない
    \item 専門用語や意図的な表現は,無視して構わない
    \item ただし,多くの指摘がある場合は,文章が読みにくい可能性がある
\end{itemize}

\subsection{latexindentによるコード整形}

このテンプレートには,\textbf{latexindent}による \LaTeX ソースコードの自動整形機能が含まれている.

\subsubsection{latexindentとは}

latexindentは,\LaTeX のソースコードを読みやすく整形するツールである.

\begin{itemize}
    \item インデント(字下げ)を自動的に調整
    \item 環境やブロックごとに適切な階層構造を整理
    \item コードの可読性を向上
    \item チーム開発時のコードスタイルの統一
\end{itemize}

\subsubsection{latexindentの実行方法}

ターミナルで以下のコマンドを実行する.

\begin{minted}{bash}
# 特定のファイルを整形
latexindent -w sections/introduction.tex

# 全てのTeXファイルを一括整形
bash bin/linter-tex-fmt.sh
\end{minted}

\begin{itemize}
    \item \texttt{-w}オプションで,元のファイルを上書き保存する
    \item 整形前のファイルは \texttt{backups/}フォルダにバックアップされる
    \item \texttt{localSettings.yaml}でインデント幅などの設定をカスタマイズできる
\end{itemize}

\subsubsection{整形の例}

整形前:

\begin{minted}{latex}
\begin{itemize}
\item 項目1
\item 項目2
\begin{itemize}
\item サブ項目
\end{itemize}
\end{itemize}
\end{minted}

整形後:

\begin{minted}{latex}
\begin{itemize}
 \item 項目1
 \item 項目2
 \begin{itemize}
 \item サブ項目
 \end{itemize}
\end{itemize}
\end{minted}

\subsubsection{注意点}

\begin{itemize}
    \item 整形後は必ずコンパイルして,意図しない変更がないか確認する
    \item 特殊な記述や意図的なレイアウトが崩れる場合がある
    \item その場合は,\texttt{localSettings.yaml}で設定を調整するか,該当箇所を手動で修正する
    \item Gitで管理している場合,整形前後の差分を確認することをお勧めする
\end{itemize}

\subsection{ファイル構成}

以下のファイル構成になっている.

\begin{itemize}
    \item \texttt{main.tex}:メインファイル(全体の構成を記述)
    \item \texttt{sections/}:各章のファイルを格納
          \begin{itemize}
              \item \texttt{title.tex}:タイトルページ
              \item \texttt{abstract.tex}:概要
              \item \texttt{introduction.tex}:序論
              \item \texttt{theory.tex}:理論
              \item \texttt{conclusion.tex}:結論
              \item \texttt{references.tex}:参考文献
              \item \texttt{acknowledgments.tex}:謝辞
          \end{itemize}
    \item \texttt{figures/}:図ファイルを格納
    \item \texttt{references.bib}:参考文献データベース
\end{itemize}

各章のファイルは \verb|\subfile{sections/ファイル名}|で読み込まれる.

\section{最後に}

卒業論文は「完璧さ」よりも,以下が重要である.

\begin{itemize}
    \item 読み手に伝わるか
    \item 論理の流れが通っているか
    \item 根拠が示されているか
\end{itemize}

\textbf{「これを他人が読んで理解できるか?」}を常に意識しながら書く.
\end{comment}

\end{document}
