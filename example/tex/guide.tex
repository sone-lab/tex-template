%! TEX root = ../readme_report.tex
\documentclass[../readme_report]{subfiles}

\begin{document}

\MyChapter{卒業論文を書くときの注意点}

\section{卒業論文とは何か}

卒業論文は,自分が行った研究を,第三者に正確に説明するための学術文書です.

\begin{itemize}
    \item 感想文やレポートではない
    \item 「頑張った」「難しかった」などの主観的記述は不要である
\end{itemize}

読んだ人が以下を理解できることが重要である.

\begin{itemize}
    \item 何を目的に
    \item 何をして
    \item どんな結果が出て
    \item それをどう解釈したのか
\end{itemize}

\section{論文全体の構成について}

基本的には,以下の流れが自然になるように構成する.

\subsection{構成の要素}

\begin{enumerate}
    \item \textbf{背景・目的}:なぜこの研究を行ったのか
    \item \textbf{方法}:何を,どのように行ったのか
    \item \textbf{結果}:実験・解析で何が得られたのか
    \item \textbf{考察}:結果が何を意味するのか
    \item \textbf{結論}:研究として何が分かったのか
\end{enumerate}

章や節の順番は,「初めて読む人が迷わないか?」を基準に考える.

\section{図・表・数式の使い方}

\subsection{図・表について}

図や表は本文中で必ず説明する.

\begin{itemize}
    \item \textbf{NG:}「下図に示す」
    \item \textbf{OK:} \Fig{fig:example}に示すように,\ldots
\end{itemize}

図や表は単体で見ても意味が分かるように作成する.以下を確認する.

\begin{itemize}
    \item 軸の名前
    \item 単位
    \item 記号の説明
\end{itemize}

「見た目がきれい」より「情報が正確」が重要である.

\subsection{数式について}

数式を出すときは,\textbf{なぜその式を使うのか}を文章で説明する.

\begin{itemize}
    \item いきなり式だけを出すのはNG
    \item 数式も,本文中で番号を使って参照する
\end{itemize}

例えば,\Eq{eq:example}のような形式で参照する.

\section{図の作成に関する注意}

\begin{itemize}
    \item ExcelやPowerPointで作った図を使っても構わない
    \item ただし,\textbf{文字が小さすぎて読めない図はNG}である
    \item 不要な色や装飾は避け,説明に必要な情報だけを残す
    \item 図は「飾り」ではなく「説明のための道具」である
\end{itemize}

\section{文章の書き方について}

\begin{itemize}
    \item 文体は \textbf{「だ・である調」}に統一する
    \item 主語と述語が対応しているか,必ず見直す
    \item 以下の表現は使いすぎないよう注意してください
          \begin{itemize}
              \item 「\(\sim \)と思われる」
              \item 「\(\sim \)と考えられる」
          \end{itemize}
    \item 事実(結果)と考察(解釈)は,はっきり区別して書く
    \item 箇条書きは便利であるが,多用せず,文章で説明することを基本とする
\end{itemize}

\section{引用・参考文献について}

\begin{itemize}
    \item 他人の研究成果や文章を使う場合は,\textbf{必ず引用}する
    \item 図や表を参考にした場合も,出典を明記する
    \item Webページも参考文献として記載が必要である
    \item 「自分の結果」と「他人の成果」が混ざらないよう注意する
\end{itemize}

本文中での引用例は,\cite{ref:example}のように記述する.

\section{再現性を意識する}

卒業論文では,\textbf{他人が同じことをすれば,同じ結果が出るか}が重要である.

以下の情報は,省略せずに具体的に書く.

\begin{itemize}
    \item 実験条件
    \item 使用した装置
    \item パラメータ設定
\end{itemize}

「一般的だから」「よく知られているから」と省略する場合は,どの文献や知識に基づくかを示す.

\section{禁止事項・注意事項}

\begin{itemize}
    \item 他人の文章や図を,出典なしで使ってはいけない
    \item コピー \&ペーストの多用は不正行為である
    \item 生成AIを使った場合も,\textbf{内容の責任は自分にある}
    \item 分からないことがあれば,\textbf{締切直前ではなく早めに相談}する
\end{itemize}

\section{\LaTeX での具体的な書き方}

このテンプレートを使用する際の,具体的な記述方法を説明する.

\subsection{図の挿入方法}

図を挿入する際は,\texttt{figure}環境を使用する.

\begin{equation}
    \text{配置オプション} = \text{[htbp]}
    \label{eq:placement}
\end{equation}

図のコード例を示します.

\begin{minted}{latex}
\begin{figure}[htbp]
 \centering
 \includegraphics[width=0.8 \linewidth]{figures/example.pdf}
 \caption{実験結果の例}
 \label{fig:example}
\end{figure}
\end{minted}

\begin{itemize}
    \item \texttt{[htbp]}は配置オプション(here, top, bottom, page)である
    \item \texttt{width=\textbackslash linewidth}で図の幅を調整(例:\texttt{0.8 \textbackslash linewidth})
    \item \texttt{\textbackslash label \{fig:label \}}でラベルを付けると,本文中で参照できる
    \item 図ファイルは \texttt{figures/}フォルダに配置する
    \item PDFまたはPNG形式の図を推奨する(EPS形式も使用可能)
\end{itemize}

本文中での図の参照:

\begin{minted}{latex}
\Fig{fig:label}に示すように,...
\end{minted}

または

\begin{minted}{latex}
\Figure{fig:label}に示すように,...
\end{minted}

\begin{itemize}
    \item \verb|\Fig{...}|は「Fig. 1」のように表示される
    \item \verb|\Figure{...}|は「Figure 1」のように表示される
\end{itemize}

\subsection{表の挿入方法}

表を挿入する際は,\texttt{table}環境を使用する.

\begin{minted}{latex}
\begin{table}[htbp]
 \centering
 \caption{実験条件}
 \label{tab:conditions}
 \begin{tabular}{lcc}
 \hline
 項目 & 値 & 単位 \\
 \hline
 温度 & 25 & \si{\celsius} \\
 圧力 & 101.3 & \si{\kilo \pascal} \\
 \hline
 \end{tabular}
\end{table}
\end{minted}

\begin{itemize}
    \item \texttt{\textbackslash caption}は表の上に配置する(図の場合は下)
    \item \texttt{\textbackslash hline}で横線を引く
    \item 列の配置は \texttt{l}(左寄せ),\texttt{c}(中央),\texttt{r}(右寄せ)で指定する
\end{itemize}

本文中での表の参照:

\begin{minted}{latex}
\Table{tab:label}に示すように,...
\end{minted}

\subsection{数式の書き方}

\subsubsection{インライン数式(文中の数式)}

本文中に \verb|$E = mc^2$|のように記述する.

\subsubsection{別行立て数式(番号付き)}

別行立て数式では \texttt{equation}環境を使用する.

本文中での式の参照:

\begin{minted}{latex}
\Eq{eq:label}より,...
\end{minted}

\verb|\Eq{...}|は「Eq. (1)」のように表示される.

\subsubsection{複数行の数式}

複数行の数式は \texttt{align}環境を使用する.

\begin{align}
    a & = b + c \\
    d & = e + f
\end{align}

\subsection{参考文献の引用方法}

このテンプレートでは \textbf{biblatex}を使用している.

\subsubsection{文献情報の登録}

\verb|references.bib|ファイルに文献情報を記述する.例を示す.

本文中での引用:

\begin{minted}{latex}
先行研究 \cite{ref:yamada2020}によると,...
\end{minted}

複数の文献を同時に引用する場合:

\begin{minted}{latex}
これらの研究 \cite{ref:yamada2020, ref:tanaka2021}では,...
\end{minted}

\subsection{章・節の作成方法}

このテンプレートでは,各章を個別のファイルに分けて記述する.

\subsubsection{章の作成}

\begin{minted}{latex}
\MyChapter{章のタイトル}
\end{minted}

\begin{itemize}
    \item 通常の \verb|\chapter{...}|の代わりに \verb|\MyChapter{...}|を使用する
    \item ヘッダーに章タイトルが表示される
\end{itemize}

\subsubsection{節・小節の作成}

\begin{itemize}
    \item \texttt{\textbackslash section \{節のタイトル \}}を使用する
    \item \texttt{\textbackslash subsection \{小節のタイトル \}}を使用する
    \item \texttt{\textbackslash subsubsection \{小小節のタイトル \}}を使用する
\end{itemize}

\subsection{特殊な記号・単位の書き方}

\subsubsection{単位の記述}

このテンプレートには \texttt{siunitx}パッケージが含まれている.

\begin{minted}{latex}
温度は \SI{25}{\celsius},速度は \SI{10}{\meter \per \second} である.
\end{minted}

\subsubsection{ベクトルの記述}

\begin{minted}{latex}
ベクトル $\bm{v}$ は...
\end{minted}

\subsection{textlintによる文章チェック}

このテンプレートには,\textbf{textlint}による日本語文章の自動チェック機能が含まれている.

\subsubsection{textlintとは}

textlintは,文章の表記ゆれや誤字,日本語の文法的な問題を自動的にチェックするツールである.

\begin{itemize}
    \item 「ですます調」と「である調」の混在をチェック
    \item 一文の長さや読点の位置をチェック
    \item 冗長な表現や不適切な表記を検出
    \item 技術文書に適した表現かどうかを検証
\end{itemize}

\subsubsection{textlintの実行方法}

ターミナルで以下のコマンドを実行する.

\begin{minted}{bash}
# 初回のみ:textlintのインストール
bash bin/install_textlint.sh

# 文章のチェック実行
bash bin/linter.sh
\end{minted}

\subsubsection{チェック結果の確認}

textlintは,問題がある箇所を以下のように報告する.

\begin{itemize}
    \item ファイル名と行番号
    \item 問題の種類(エラーまたは警告)
    \item 具体的な修正提案
\end{itemize}

\subsubsection{注意点}

\begin{itemize}
    \item textlintの指摘は \textbf{あくまで参考}である
    \item すべての指摘を修正する必要はない
    \item 専門用語や意図的な表現は,無視して構わない
    \item ただし,多くの指摘がある場合は,文章が読みにくい可能性がある
\end{itemize}

\subsection{latexindentによるコード整形}

このテンプレートには,\textbf{latexindent}による \LaTeX ソースコードの自動整形機能が含まれている.

\subsubsection{latexindentとは}

latexindentは,\LaTeX のソースコードを読みやすく整形するツールである.

\begin{itemize}
    \item インデント(字下げ)を自動的に調整
    \item 環境やブロックごとに適切な階層構造を整理
    \item コードの可読性を向上
    \item チーム開発時のコードスタイルの統一
\end{itemize}

\subsubsection{latexindentの実行方法}

ターミナルで以下のコマンドを実行する.

\begin{minted}{bash}
# 特定のファイルを整形
latexindent -w sections/introduction.tex

# 全てのTeXファイルを一括整形
bash bin/linter-tex-fmt.sh
\end{minted}

\begin{itemize}
    \item \texttt{-w}オプションで,元のファイルを上書き保存する
    \item 整形前のファイルは \texttt{backups/}フォルダにバックアップされる
    \item \texttt{localSettings.yaml}でインデント幅などの設定をカスタマイズできる
\end{itemize}

\subsubsection{整形の例}

整形前:

\begin{minted}{latex}
\begin{itemize}
\item 項目1
\item 項目2
\begin{itemize}
\item サブ項目
\end{itemize}
\end{itemize}
\end{minted}

整形後:

\begin{minted}{latex}
\begin{itemize}
 \item 項目1
 \item 項目2
 \begin{itemize}
 \item サブ項目
 \end{itemize}
\end{itemize}
\end{minted}

\subsubsection{注意点}

\begin{itemize}
    \item 整形後は必ずコンパイルして,意図しない変更がないか確認する
    \item 特殊な記述や意図的なレイアウトが崩れる場合がある
    \item その場合は,\texttt{localSettings.yaml}で設定を調整するか,該当箇所を手動で修正する
    \item Gitで管理している場合,整形前後の差分を確認することをお勧めする
\end{itemize}

\subsection{ファイル構成}

以下のファイル構成になっている.

\begin{itemize}
    \item \texttt{main.tex}:メインファイル(全体の構成を記述)
    \item \texttt{sections/}:各章のファイルを格納
          \begin{itemize}
              \item \texttt{title.tex}:タイトルページ
              \item \texttt{abstract.tex}:概要
              \item \texttt{introduction.tex}:序論
              \item \texttt{theory.tex}:理論
              \item \texttt{conclusion.tex}:結論
              \item \texttt{references.tex}:参考文献
              \item \texttt{acknowledgments.tex}:謝辞
          \end{itemize}
    \item \texttt{figures/}:図ファイルを格納
    \item \texttt{references.bib}:参考文献データベース
\end{itemize}

各章のファイルは \verb|\subfile{sections/ファイル名}|で読み込まれる.

\section{最後に}

卒業論文は「完璧さ」よりも,以下が重要である.

\begin{itemize}
    \item 読み手に伝わるか
    \item 論理の流れが通っているか
    \item 根拠が示されているか
\end{itemize}

\textbf{「これを他人が読んで理解できるか?」}を常に意識しながら書く.

\end{document}
