%! TEX root = ../../readme_report.tex
\documentclass[readme_report]{subfiles}

\begin{document}
\MyChapter{卒業論文を書くときの注意点}

\section{卒業論文とは何か}

卒業論文は,自分が行った研究を,第三者に正確に説明するための学術文書である.
従来の感想文やレポートとは異なり,主観的な記述は避け,客観的にかつ再現性 \footnote{あなたのことを知らない第三者がこの文章を読むだけで同じ実験を行えること}を重視した内容が求められる.
このため,読んだ人がこの研究は何を目的に行われ,どのような方法で実施され,どのような結果が得られ,それをどのように解釈したのかを理解できるように書く必要がある.

\section{論文全体の構成について}

基本的には,以下の流れが自然になるように構成する.

\subsection{構成の要素}

\begin{enumerate}
    \item \textbf{背景・目的}:なぜこの研究をしたのか
    \item \textbf{方法}:何を,どのように行ったのか
    \item \textbf{結果}:実験・解析で何が得られたのか
    \item \textbf{考察}:結果が何を意味するのか
    \item \textbf{結論}:研究として何が分かったのか
\end{enumerate}

章や節の順番は,ある程度の共通認識があるが,研究内容によっては構成が異なる場合もある.
しかしながら,上記の要素が論理的に繋がるように注意することが重要である.

論文の書き方については,たくさんの文献 \cite{cite:HuangLaiKangSi2019,cite:ShiHeiGui2012}があるので,参考にするとよい.

\section{図・表・数式の使い方}

\subsection{図・表について}

図や表は本文中で必ず説明する必要があり,文中では番号を使って参照する.
この番号は手入力を行わない.\footnote{途中で図表の順番の入れ替え,追加,削除を行うことはよくあるため,自動的に番号が振られる仕組みを使わなければならない.自分はB4のときに酷い目にあった経験がある.}
「下図に示す」のような表現は避け,「\Fig{fig:fig_example}に示すように」のように記述する.
また,図や表は単体で見ても意味が分かるように作成する.
軸の名前,単位,記号,凡例,その図の意味などをキャプションに記述する.
そのためにキャプションが長くなっても構わない.
これらのルールは,本来文章と図の位置関係は著者側に制御できず,印刷時に図が本文から離れてしまう可能性があるためである.


\subsection{数式について}

数式を用いるときは,その数式の意味や役割を本文中で説明する.
また,変数や記号の定義や単位系の説明も必要である.
図表と同様に,数式も本文中で番号を使って参照する.
例えば,\Eq{eq:eq_example}のような形式で参照する.

\section{図の作成に関する注意}

図は,研究の結果を効果的に伝えるための重要な要素である.
この良し悪しが論文の評価を大きく左右することもある.

図は,ExcelやPowerPointで作った図を使っても良いが,draw.ioやPythonなどのツールを用いても良い.
特に,大規模なデータをエクセルで処理するのは非効率であり,PythonやRなどのプログラミング言語を用いることが推奨される.

論文に用いる図は,基本的にベクタ形式のPDFやSVGで保存することが望ましい.
ラスタ形式であるJPEGやPNGは,解像度が低くなりがちであり,印刷時にぼやけてしまう可能性があるためである.
また,印刷時に拡大しなければ読めないような文字サイズは避け,余計な装飾や色使いも控える.

\section{文章の書き方について}

文章は,敬体(ですます調)ではなく,常体(である調)で書く.
特によくある良くない例として,主語と述語の不一致がある,文章が長過ぎる,曖昧な表現が多い,などが挙げられる.
「\(\sim \)と思われる」や「\(\sim \)と考えられる」などの表現は,客観性を欠くため,使用を避けるべきである.
事実と考察は明確に区別し,箇条書きなできるだけ避け,文章で説明することが望ましい.

\section{引用・参考文献について}
\label{sec:citation}

他人の著作物を使用する場合は,必ず引用を行う必要があり,これは「引用の要件」を満たす必要がある.

\begin{itembox}[l]{引用の要件 \cite{cite:aca2024}}
    \begin{itemize}
        \item 公表された著作物であること(未公表の著作物は×)
        \item 公正な慣行に合致すること(引用の「必然性」があること,引用する部分が「明確に区別」されること)
        \item 引用の目的上「正当な範囲内」であること(自分の著作物と他人の著作物との間に妥当な「主従関係」があること, 引用する分量が必要最小限度の範囲内であること)
        \item 「出所の明示」が必要
    \end{itemize}
\end{itembox}

他人の著作物を引用し,引用の要件を満たしていない場合,盗用(剽窃)と見なされ,不正行為となる可能性がある \cite{cite:KanagawaUniversity2026a}.

本文中での引用例は,\cite{cite:KanagawaUniversity2026a}のように記述する.

\section{再現性を意識する}

卒業論文では,他人が同じことをすれば,同じ結果が出るかが重要である.
ある研究を引き継ぎながら進めていく形式であるため,あとから見返す人(主に後輩)が同じ実験を再現できるように書く必要がある.
このため,実験条件,使用した装置,パラメータ設定などを具体的に記述する.
「一般的だから」「よく知られているから」と省略する場合は,どの文献や知識に基づくかを示す.

\section{禁止事項・注意事項}

\ref{sec:citation}で述べた通り,他人の文章や図を,出典なしで使ってはいけない.
いわゆる「生成AI」と称されるLLM(大規模言語モデル)を使用することは厳密に禁止することはしないが,使用した責任は自分にあることを認識する必要がある.
大学の規則 \cite{cite:ShenNaiChuanDaXue}をよく確認すること.
特に研究の内容は機密情報であるため,LLMに入力してはならない.
論文化や特許出願の妨げになる可能性があるためである.

また,わからないことがあれば,締切直前ではなく早めに相談する \footnote{タイミングが悪いと締め切りまでに返事できなかったりします}こと.

\end{document}
